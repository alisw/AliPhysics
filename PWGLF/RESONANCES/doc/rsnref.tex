\documentclass[12pt,a4paper]{article}

\usepackage{vmargin}
\usepackage{listings}

\renewcommand{\rmdefault}{cmss}
\newcommand{\inlinecode}[2]{\mbox{{\bf #1(}#2{\bf)}}}
\newcommand{\classb}[1]{\mbox{{\bf #1}}}
\newcommand{\classi}[1]{\mbox{{\em #1}}}

\setpapersize{A4}
\setmarginsrb{10mm}{10mm}{10mm}{10mm}{0mm}{10mm}{0mm}{10mm}

\title{ALICE resonance analysis package - Reference Guide}
\author{A. Pulvirenti}
\date{\tt alberto.pulvirenti@ct.infn.it}

\begin{document}
\lstset{language=C++}
\lstset{basicstyle=\tiny}

\maketitle

\section{Introduction}

This document contains a reference guide for all classes in the package, which complements what is explained
in the online reference class guide which is automatically available from AliRoot code pages.

\section{AliRsnDaughter}

This object is used to have a unique access point to all informations coming from candidate daughters of a resonance.
Such a candidate can be of any object type which can be related to a single particle: track, V0, cascade.
In fact, \classb{AliRsnDaughter} contains the enumeration \classb{ERefType} which tells what kind of object is linked to it.
This class does not contain any real processing function, because it just provides a pointer to an \classi{AliVParticle} object
and eventually to its related MC particle, but it does not execute any specific computation.

Unique exception is the \inlinecode{SetMass}{Double_t} method, which is used to assign a mass hypothesis to it.
This is needed since the 4-momentum of the related particle is stored into a \classi{TLorentzVector} datamember, and it
is filled by this function, which adopts the vector momentum from the referenced \classi{AliVParticle} and uses that mass
to compute the fourth component of this 4-momentum. 
The user is not allowed to set directly the vector momentum, since it must always be coherent with that of the referenced object.


\end{document}
