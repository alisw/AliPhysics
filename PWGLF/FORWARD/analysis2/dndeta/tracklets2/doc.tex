\documentclass{article}

\usepackage[margin=2cm,a4paper]{geometry}
\usepackage{amsmath}
\usepackage{amstext}
\usepackage{hyperref}
\newcommand\dndeta{% 
  \ensuremath%
  \mathrm{d}N_{\mathrm{ch}}/\mathrm{d}\eta}
\title{$\dndeta$ from SPD tracklets}
\author{Christian Holm Christensen}
\date{\today}
\begin{document}
\maketitle 

\section{Fundamentals}

The fundamental formula used in the analysis of SPD tracklets is 
\begin{align}
  \label{eq:fundamental}
  R &= \frac{P}{(1-\beta^\prime)M^\prime}\left(1-\beta\right)M\quad,
\end{align}
where
\begin{itemize}
\item[$R$] is the result
\item[$P$] is the distribution of generated primary particles in
  simulations. 
\item[$M^\prime$] is distribution of measured (reconstructed) tracklets
  from simulated data 
\item[$\beta^\prime$] is the relative contribution of background to
  $M^\prime$ 
\item[$M$] is the distribution of measured (reconstructed) tracklets
  from real data
\item[$\beta$] is the relative contribution of background to $M$ 
\end{itemize}

Here, all variables are two-dimensional distributions over $\eta$ and
$\mathrm{IP}_z$ of number of tracklets (or particles for $P$) for a
given centrality or other event selection. 

The background $\beta$ (for both simulated and real data) can be
estimated in two ways: 

\begin{description}
\item[Injection] By injecting fake clusters into the events and
  reconstructing the event.
\item[Combinatorics] By inspecting the mothers of the two constituent
  clusters. 
\end{description}

Strictly speaking, the latter method is only possible on simulated
data, but the estimate from simulated data can be used as an estimate
for real data. 

\section{Injection}

When estimating the background from injected clusters, the analysis
builds the distribution $I$ using several reconstruction passed on the
same event.  In each pass, a real cluster is removed and a new fake
cluster, similar to the removed one, is placed elsewhere in the SPD.
The distribution $I$ therefore has a large number of `fake'
tracklets.  

For both the tracklets in the distribution $I$ from injected clusters,
and from the tracklets of the measured distribution $M$, we can
determine the distribution $f(\Delta)$ of the quality parameter
$\Delta$ given by  
\begin{align}
  \label{eq:Delta}
  \Delta 
  &= 
    \left(\frac{\Delta\varphi-\delta_{\varphi}}{\sigma_{\varphi}}\right)^2+
    \left(\frac{\Delta\vartheta\sin^{-2}(\vartheta)}{\sigma_{\vartheta}}\right)^2
    \quad,
\end{align}
where $\Delta\varphi$ and $\Delta\vartheta$ are the residuals with
respect to the expected tracklet trajectory, and $\sigma_{\varphi}$
and $\sigma_{\vartheta}$ are the resolutions in the two angles. 

Signal tracks are expected to have a
$\Delta<\Delta_{\mathrm{cut}}=1.5$, while background tracks will have
a $\Delta>\min\Delta_{\mathrm{tail}}=5$.  We can therefore extract the
background $B$ from the distribution $I$ by matching the tails of the
$\Delta_I$ distribution to that of $\Delta_M$, or more precisely 
\begin{align}
  \label{eq:scaledI}
  B 
  &= \frac{\int_{\min\Delta_{\mathrm{tail}}}^{\infty}\mathrm{d}\Delta
    f_M(\Delta)}{
    \int_{\min\Delta_{\mathrm{tail}}}^{\infty}\mathrm{d}\Delta
    f_I(\Delta)}I\quad.
\end{align}

We then calculate the relative contribution of background particle
$\beta$ as 
\begin{align}
  \label{eq:betaI}
  (1-\beta)M &= \left(1-\frac{B}{M}\right)M = M-B = S\quad.
\end{align}
Here, $S$ is the signal distribution.  For simulated data, we simply
replace the un-primed distributions with the primed equivalent
\begin{align*}
  % \label{eq:betaIMC}
  (1-\beta^\prime)M^\prime 
  &=
    \left(1-\frac{B^\prime}{M^\prime}\right)M^\prime 
    = M^\prime-B^\prime = S^\prime\quad,
\end{align*}
where $B^\prime$ is calculated from $I^\prime$ as given in
\eqref{eq:scaledI} where we have injected clusters into a simulated
event. 

\section{Combinatorics}

By selecting tracklets for which the two constituent clusters come
from different tracks in the simulated data, we can estimate the
(combinatorial) background $C^\prime$ directly.  We can therefore calculate
the signal distribution in simulated data as 
\begin{align}
  \label{eq:betaCMC}
  (1-\beta^\prime)M^\prime 
  &= \left(1-\frac{C^\prime}{M^\prime}\right)M^\prime =
    M^\prime - C^\prime = S^\prime\quad,
\end{align}
where we again have identified the signal distribution $S^\prime$ for
simulated data. 

We cannot estimate $C$ for real data, so instead, we assume the same
combinatorial background in real data, with a scaling factor, and for
real data, we get 
\begin{align}
  \label{eq:betaC}
  (1-\beta)M 
  &= (1-k\beta^\prime)M 
    = \left(1-k\frac{C^\prime}{M^\prime}\right)M = S\quad.
\end{align}
Note, in this case, the expression does not reduce further.  The
constant $k=1.3$ expresses the additional amount of combinatorial
background seen in real as opposed to simulated data. 

\section{Putting it together} 

We can put all the above together to form 4 different ways of
calculating $R$ 

\begin{align}
  \label{eq:methods}
  R &=
      \begin{cases}
        \frac{P}{M^\prime-C^\prime}\left(1-k\frac{C^\prime}{M^\prime}\right)M 
        & \text{Combinatorial---Combinatorial}\\
        \frac{P}{M^\prime-C^\prime}(M-B) 
        & \text{Combinatorial---Injection}\\
        \frac{P}{M^\prime-B^\prime}\left(1-k\frac{C^\prime}{M^\prime}\right)M 
        & \text{Injection---Combinatorial}\\
        \frac{P}{M^\prime-B^\prime}(M-B)
        & \text{Injection---Injection}\quad.
      \end{cases}
\end{align}
Note, in the first case (Combinatorial---Combinatoral), if $k=1$, the
equation simplifies to 
\begin{align*}
  R &= \frac{P}{M^\prime}M\quad,
\end{align*}
meaning we can identify 
\begin{align}
  \label{eq:alpha}
  \alpha &= \frac{P}{(1-\beta^\prime)M^\prime}
\end{align}
as an effective correction from the measurements to the number of
primary particles. 
\end{document}
