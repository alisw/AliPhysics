\begin{abstract}
  \noindent
  This note details the analysis for \ndndeta{} over the widest
  possible $\eta$ range possible with \ALICE{} (and \LHC{} for that
  matter).  The analysis uses tracklets from the \SPD{} and signals
  from the \FMD{}.  All possible collision systems and energies are
  analysed.  For \ppCol{} \ndndeta{} is determined for the three
  trigger classes \INEL{}, \INELGt{}, and \NSD{}.  For \PbPbCol{},
  \pPbCol{}, and \PbpCol{} \ndndeta{} is determined as a function of
  centrality. Secondary particles produced in surrounding
  material and decays are removed from the \FMD{} signal using an
  \emph{empirical} correction based on results
  from \cite{Abbas:2013bpa}. \\

  \noindent 
  The focus in this note is on the results from \PbPbCol{} at
  $\usNN{PbPb}{2760}$ over all centrality bins.  Work--in--progress
  results for other systems are also shown here, but they are not
  finalised.  Separate follow--up notes will handle those systems. \\

  \noindent
  This notes details the forward (\FMD{}) part of the analysis only.
  A separate note on the \SPD{} tracklet analysis exists elsewhere
  \cite{spdnote}.  The content of this note is in many places the same
  as in an older analysis note \cite{oldnote}, but is otherwise
  independent and can be read on its own.\\

  \noindent
  {\small TWiki page: \url{http://cern.ch/go/fm8C}}
\end{abstract}

%% Local Variables:
%%   TeX-master: "PWGLF_Forward_analysis_note.tex"
%%   ispell-dictionary: "british"
%% End:
