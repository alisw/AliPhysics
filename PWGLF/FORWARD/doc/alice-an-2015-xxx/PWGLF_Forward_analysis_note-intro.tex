%% === The Introduction ==============================================
\section{Introduction}

This note describes the steps performed in the analysis of the
charged particle multiplicity in the forward pseudo--rapidity regions
with the \FMD{} detector \cite{FWD:2004mz,cholm:2009}. 

\paragraph{The \FMD{}}
The \FMD{} is organised in 3 \emph{sub--detectors} \FMD{1}, \FMD{2},
and \FMD{3}, each consisting of 1 (\FMD{1}) or 2 (\FMD{2} and~3)
\emph{rings}.  The rings fall into two types: \emph{Inner} or
\emph{outer} rings.  Each ring is in turn azimuthally divided into
\emph{sectors}, and each sector is radially divided into
\emph{strips}.  The number of sectors, strips, as well as the nominal
$\eta$ coverage for each ring are given in
\tablename~\ref{tab:fmd:overview}.

\begin{table}[htbp]
  \begin{center}
    \caption{Physical dimensions of Si segments and strips.}
    \label{tab:fmd:overview}
    \vglue0.2cm
    \begin{tabular}{|c|cc|crcl|rcl|}
      \hline
      \headColor%
      \textbf{Sub--detector/} &
      \textbf{Azimuthal}&
      \textbf{Radial} &
      $z$ &
      \multicolumn{3}{c|}{\textbf{$r$}} &
      \multicolumn{3}{c|}{\textbf{$\eta$}} \\ 
      \rowcolor{alicered!25!white}
      \textbf{Ring}&  
      \textbf{sectors} &
      \textbf{strips} & 
      \textbf{[cm]} &
      \multicolumn{3}{c|}{\textbf{range [cm]}} &
      \multicolumn{3}{c|}{\textbf{coverage}} \\
      \hline
      FMD1i & 20& 512& \phantom{-}320  &  4.2&--&17.2& 3.68&--&\phantom{-}5.03\\
      \altRowColor{}%
      FMD2i & 20& 512&  \phantom{-}83.4&  4.2&--&17.2& 2.28&--&\phantom{-}3.68\\
      FMD2o & 40& 256&  \phantom{-}75.2& 15.4&--&28.4& 1.70&--&\phantom{-}2.29\\
      \altRowColor{}%
      FMD3i & 20& 512& -75.2&  4.2&--&17.2&-2.29&--&-1.70\\
      FMD3o & 40& 256& -83.4& 15.4&--&28.4&-3.40&--&-2.01\\
      \hline
    \end{tabular}
  \end{center}
\end{table}

\paragraph{Event Summary Data} 
The \FMD{} \ESD{} object contains the scaled energy deposited
$\Delta_t\equiv\Delta /\DeltaMip$ for each of the 51\,200 strips.  This
is determined in the reconstruction pass.  The scaling to $\DeltaMip$
is done using calibration factors extracted in designated calibration
runs.  In these runs, the front--end electronics is pulsed with an
increasing known pulse size, and the conversion factor from ADC counts
to $\DeltaMip$ is determined \cite{cholm:2009}.

The \SPD{} is used for determination of the position of the primary
interaction point.

\paragraph{Analysis steps} 
The analysis is performed as a two--step process.  
\begin{enumerate}
\item The Event--Summary--Data (\ESD{}) is processed event--by--event.
  On each event, the data is passed through a number of algorithms,
  and $\dndetadphi$ for each event is output to an
  Analysis--Object--Data (\AOD{}) tree.
\item The \AOD{} data is read in and the sub--sample of the data under
  investigation is selected. For \ppCol{} we select of for example
  \INEL{}, \INELONE{}, and \NSD{}\footnote{Recently, capability to
    select on other estimators relevant for high--multiplicity studies
    have been implemented.}. For \PbPbCol{}, \pPbCol{}, and \PbpCol{}
  we select on centrality estimators.  The $\dndetadphi$ histogram
  read for the selected events is used to build up $\ndndeta$.
\end{enumerate}

\secref{sec:gen_aod} of this note details the first step.  It gives
the details of how to build the event--by--event $\dndetadphi$.
\secref{sec:ana_aod} gives the details of how the final $\ndndeta$ is
built from the event--by--event $\dndetadphi$.  \secref{sec:ana_aod}
also details the normalisation as well as the systematic
uncertainties.  Finally, in \secref{sec:results} the result plots is
shown together with the particular details of each result. 

In \secref{app:nomen} is an overview of the nomenclature used in
this document.
%% Local Variables:
%%   TeX-master: "PWGLF_Forward_analysis_note.tex"
%%   ispell-dictionary: "british"
%% End:
