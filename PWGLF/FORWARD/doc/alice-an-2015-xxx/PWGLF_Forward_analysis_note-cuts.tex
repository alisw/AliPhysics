%% === Section on how cuts are defined ===============================
\section{Cuts}
\label{sec:cuts}


Cuts on the energy loss can be defined in 6 ways all with the single
parameter $X$:
\begin{description}
\item[\emph{Fixed value} $c=X$]
  The cut is defined as a hard cut at a particular value $X$.  This
  type of cut is typically used in the sharing filter (see
  \secref{sec:sub:sharing_filter}) for the lower cut.
\item[\emph{Fit range}]
  The cut $c$ is set to the lower bound of the fit range of the energy
  loss fits (see \secref{sec:fits}).  This type of cut is deprecated.
\item[\emph{Fraction of $\Delta_p$} $c=X\Delta_p$] The cut is defined
  as some fraction $X<1$ of the most probably value of obtained in
  the energy loss fits.  Cuts defined this way are not used in the
  analysis. 
\item[\emph{Landau width} $c=\Delta_p-X\xi$] The cut is defined as
  some number (not necessarily integer) Landau widths below the most
  probably value obtained in the energy loss fits. Cuts defined this
  way are not used in the analysis. 
\item[\emph{Landau and Gauss width} $c=\Delta_p - X(\xi+\sigma)$] The
  cut is defined as some number (not necessarily integer) Landau and
  Gaussian widths below the most probably value obtained in the energy
  loss fits. This type of cuts is typically used in the sharing filter
  (see \secref{sec:sub:sharing_filter}) for the high cut, and in the
  density calculations (see \secref{sec:sub:density_calculator}) for
  the `hit' threshold.
\item[\emph{Probability} $c:P(x<c)<X$] The cut $c$ is defined as
  the largest value for which 
  \begin{align*}
    P(x<c) &= \int_0^c\text{d}x f(x;\Delta_p;\xi;\sigma) < X\quad.
  \end{align*}
  Although mathematically sound, it is slow to compute and adds very
  little in terms of signal quality over the \emph{Landau and Gauss
    width}  cut type.  See also \figref{fig:cut:prob:rel} for the
  correspondance of the two method. It is not used in the analysis.
\end{description}
In the above $\Delta_p,\xi,\sigma,$ and $\mathbf{a}$ refers to the
parameters obtained in the energy loss fits (see \secref{sec:fits}),
and $f$ is defined in \eqref{eq:f}.

\begin{figure}[h!tbp]
  \centering
  \figinput[.8\linewidth]{prob_rel}
  \caption{Approximate relation between $X$ value chosen for the
    \emph{Landau Gauss Width} type cut ($c=\Delta_p - X(\xi+\sigma)$),
    and the $X$ of the \emph{Probability} cut ($c:P(x<c)<X$).  The
    dashed lines indicate the interesting region.  The chosen value of
    $X=1$ for the \emph{Landau Gauss Width} type cut corresponds to a
    rejection probability of roughly 3.5\%}
  \label{fig:cut:prob:rel}
\end{figure}

\tabref{tab:cuts:used} summarises the cut methods and corresponding
values of the cut parameter $X$ used in the analysis. 

\begin{table}[h!tbp]
  \centering
  \caption{The cut definitions and values used in the analysis.}
  \label{tab:cuts:used}
  \begin{tabular}[T]{|c|ll|l|c|}
    \hline 
    \headColor
    \textbf{Name} 
    & \multicolumn{2}{l|}{\textbf{Description}}
    & \textbf{Method} 
    & $X$ \\
    \hline 
    \DataSty{l} 
    & Low sharing cut
    & \secref{sec:sub:sharing_filter} 
    & $c=X$
    & 0.15\\
    \altRowColor 
    \DataSty{h} 
    & High sharing cut
    & \secref{sec:sub:sharing_filter} 
    & $c=\Delta_p - X(\xi+\sigma)$
    & 1\\
    \hline
    $T(\eta_t)$
    & Hit threshold
    & \secref{sec:sub:density_calculator}
    & $c=\Delta_p - X(\xi+\sigma)$
    & 1\\
    \hline
  \end{tabular}
\end{table}

\subsection{Considerations on choosing cut values}

\begin{description}
\item[\DataSty{l} --- low sharing cut:]  This cut removes pedestal
  remnants left in the data after the on--line pedestal subtraction.
  If $X$ is set too low, then the data is polluted by electronics
  noise.  If it is set too high, then we remove possibly interesting
  hits.  The optimum value was chosen by varying $X$ between 0.12 and
  0.18. 
\item[\DataSty{h} --- high sharing cut:]  This cut defines when we
  consider a signal in a given strip $t$ as corresponding to the
  traversal of a single particle entirely within the strips boundaries
  i.e., the particle did not deposit energy in neighbouring strips.
  If $X$ is set too low (corresponding to a higher value of cut), then
  we will start to merge signals from two isolated particles that
  traverse neighbouring strips.  If $X$ is set too high, we will fail
  to merge signals distributed over more than 1 strip, and we loose
  hits.  The optimum value was chosen by varying $X$ between 0.8 and
  1.5. 
\item[$T(\eta_t)$ --- hit threshold:] This cut discriminates between a
  \emph{hit} or \emph{empty} strip.  If $X$ is set too low,
  corresponding to a higher cut value, we loose signals, and the
  correspondence between the two methods of estimating the inclusive
  number of charged particles (see \secref{sec:sub:sub:eloss_fits} and
  \secref{sec:sub:sub:poisson}) is lost.  Conversely, if $X$ is set
  too high, we will overestimate the inclusive number of charged
  particles. Note, $T(\eta_t)$ should always be equal to or larger
  than \DataSty{h}, or equivalently $X_T\le X_{\DataSty{h}}$.  The
  optimum value was chosen by varying $X$ from 0.8 up to the chosen
  value of $X_{\DataSty{h}}$.
\end{description}

\figref{fig:cuts:syserr} shows the effect of varying the cut
parameters $X$ on the inclusive (primary and secondary) $\ndndeta$ for
minimum bias \PbPbCol{} events at $\usNN{PbPb}{2760}$.  For the cuts
relating to merging of shared signals, we see that the variations are
all within $1\%$ (except pathological edges) and we therefor adopt a
$1\%$ systematic uncertainty from this.  For the inclusive $\Nch$
calculation we get a systematic uncertainty of ${}_{-2}^{+1}\%$.

\begin{figure}[h!tbp]
  \centering
  \figinput{cut_syserr.png}
  \caption{Systematic uncertainty from variation of cuts.  Top shows
    the inclusive $\ndndeta$ distributions for minimum bias \PbPbCol{}
    for different values of $X_{\DataSty{l}}$, $X_{\DataSty{h}}$, and
    $X_T$.  Closed markers corresponds to variations for the sharing
    filter, while open markers correspond to variations in the
    inclusive charge particle calculations.  The bottom shows the
    ratio of the inclusive $\ndndeta$ for each cut set to the
    inclusive $\ndndeta$ using the default cut values. }
  \label{fig:cuts:syserr}
\end{figure}



%% Local Variables:
%%   TeX-master: "PWGLF_Forward_analysis_note.tex"
%%   ispell-dictionary: "british"
%% End:
