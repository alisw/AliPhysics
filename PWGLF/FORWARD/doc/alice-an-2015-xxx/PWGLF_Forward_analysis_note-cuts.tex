%% === Section on how cuts are defined ===============================
\section{Cuts}
\label{sec:cuts}

\FIXME{Possibly some plots}
Cuts on the energy loss can be defined in 6 ways all with the single
parameter $X$:
\begin{description}
\item[\emph{Fixed value} $c=X$]
  The cut is defined as a hard cut at a particular value $X$.  This
  type of cut is typically used in the sharing filter (see
  \secref{sec:sub:sharing_filter}) for the lower cut.
\item[\emph{Fit range}]
  The cut $c$ is set to the lower bound of the fit range of the energy
  loss fits (see \secref{sec:fits}).  This type of cut is deprecated.
\item[\emph{Fraction of $\Delta_p$} $c=X\Delta_p$] The cut is defined
  as some fraction $X<1$ of the most probably value of obtained in
  the energy loss fits.  Cuts defined this way are not used in the
  analysis. 
\item[\emph{Landau width} $c=\Delta_p-X\xi$] The cut is defined as
  some number (not necessarily integer) Landau widths below the most
  probably value obtained in the energy loss fits. Cuts defined this
  way are not used in the analysis. 
\item[\emph{Landau and Gauss width} $c=\Delta_p - X(\xi+\sigma)$] The
  cut is defined as some number (not necessarily integer) Landau and
  Gaussian widths below the most probably value obtained in the energy
  loss fits. This type of cuts is typically used in the sharing filter
  (see \secref{sec:sub:sharing_filter}) for the high cut, and in the
  density calculations (see \secref{sec:sub:density_calculator}) for
  the `hit' threshold.
\item[\emph{Probability} $c:P(x<c)<X$] The cut $c$ is defined as
  the largest value for which 
  \begin{align*}
    P(x<c) &= \int_0^c\text{d}x f(x;\Delta_p;\xi;\sigma) < X\quad.
  \end{align*}
  Although mathematically sound, it is slow to compute and adds very
  little in terms of signal quality over the \emph{Landau and Gauss
    width}  cut type.  It is not used in the analysis.
\end{description}
In the above $\Delta_p,\xi,\sigma,$ and $\mathbf{a}$ refers to the
parameters obtained in the energy loss fits (see \secref{sec:fits}),
and $f$ is defined in \eqref{eq:f}.

%% Local Variables:
%%   TeX-master: "PWGLF_Forward_analysis_note.tex"
%%   ispell-dictionary: "british"
%% End:
