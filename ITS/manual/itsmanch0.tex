\chapter*{Introduction}
\section*{Purpose of this document}

This document is intended to both explain who to use the ALICE simulation and
reconstruction code with respect to or using the ITS detector as the
examples. This document is also to explain how to add new code to the ITS
simulation and reconstruction library, and who the existing ITS simulation and
reconstruction code works. All comments from every user is greatly encouraged.

\section*{How to Run AliRoot}

At this point, we will assume that aliroot has been compiled and all the
necessary environment variables have been defined, including your path. We will
assume you are in an appropriate directory, for example \texttt{/data} or some
such thing.

to start with type, at your shell prompt, aliroot to start the program.

\begin{verbatim}
# aliroot
Constant Field Map1 created: map= 1, factor= 1.000000
  *******************************************
  *                                         *
  *        W E L C O M E  to  R O O T       *
  *                                         *
  *   Version   2.26/00  10 November 2000   *
  *                                         *
  *  You are welcome to visit our Web site  *
  *          http://root.cern.ch            *
  *                                         *
  *******************************************

FreeType Engine v1.x used to render TrueType fonts.
Compiled with thread support.

CINT/ROOT C/C++ Interpreter version 5.14.58, Oct 24 2000
Type ? for help. Commands must be C++ statements.
Enclose multiple statements between { }.

WELCOME to ALICE

root [0]
\end{verbatim}

This will initialize ROOT and load all of the ALICE libraries. At this point you
can do anything you can do in ROOT in addition you have access to every class
defined in the ALICE libraries, including the ALICE global variables. One very
useful ALICE global variable is \texttt{gAlice} which is of type
\texttt{AliRun}. Of the many function defined in the class \texttt{AliRun} are
\texttt{Init(const char* setup="Config.C")} and 
\texttt{Run(const char* setup="Config.C")}. \texttt{Run} both executes
\texttt{Init} and starts executing an ALICE detector simulation. Both of these
functions functions load and execute a configuration file. By default this
configuration file is called \texttt{Config.C}. If such a file exists in your
local directory, for example \texttt{/data}, or if there is no such file in
your local directory it will execute the file
\texttt{\$ALICE\_ROOT/macros/Config.C}. Some other configuration file can be
run simplely by entering that file's name as the argument, for example
\texttt{Init("MyConfig.C")} or \texttt{Run("MyConfig.C")} where
\texttt{MyConfig.C} is either in your local directory (\texttt{/data}) or in
\texttt{\$ALICE\_ROOT/macros}. Of course the full path of \texttt{MyConfig.C}
can be used.

Now lets assume you just want to one event using the standard \texttt{Config.C}
file. This is done simply as
\begin{verbatim}
root [0] gAlice->Run()
Warning in <AliRun::SetField>: Invalid magnetic field flag:  -999; Helix trackin
g chosen instead

Warning in <AliFRAMEv1::ReadEuclidMedia>: file: $(ALICE_ROOT)/Euclid/frame.tme i
s now read in

Warning in <AliPIPEv0::ReadEuclidMedia>: file: $(ALICE_ROOT)/Euclid/pipe.tme is 
now read in

Warning in <AliITSv5::ReadEuclidMedia>: file: /home/CERN/aliroot/dev/Euclid/ITSg
eometry_5.tme is now read in


 MZSTOR.  ZEBRA table base TAB(0) in /MZCC/ at adr   281557647    10C83A8F HEX

 MZSTOR.  Initialize Store  0  in /GCBANK/
          with Store/Table at absolute adrs    33632021   281557647
                                        HEX     2012F15    10C83A8F
                                        HEX    F138F25A           0
                              relative adrs  -247926182           0
          with     1 Str. in     2 Links in   5300 Low words in 2999970 words.
          This store has a fence of   16 words.
\end{verbatim}
\vdots
lots more messages

\vdots
\begin{verbatim}
 TOC1      0.171%; TSSW      0.001%; TSWC      0.083%; TSCE      0.000%; TWES      0.001%;
 TSWB      0.009%; TPEL      2.240%; TPMW      0.901%; TPEW      0.128%; TESR      0.000%;
 TESB      0.116%; TPLS      0.019%; TPUS      0.023%; TPSS      0.000%; THVM      0.012%;
 TPSR      0.052%; THVL      0.025%; FLTA      0.003%; FLTB      0.000%; FLTC      0.005%;
 FMYA      0.016%; FMYB      0.014%; FMYC      0.019%; FPLA      0.023%; FPLB      0.080%;
 FPLC      0.025%; FSTR      0.138%; FNSF      0.012%; FMYX      0.002%; FGRL      0.004%;
 FPAD      0.020%; FPEA      0.039%; FPEB      0.007%; FPEC      0.062%; FECA      0.058%;
 FECB      0.058%; FECC      0.053%; FWAA      0.096%; FWAB      0.148%; FWAC      0.073%;
 FBPA      0.057%; FBPB      0.018%; FBPC      0.085%; UAFI      0.109%; UAFM      0.018%;
 UAFO      0.044%; UAII      0.001%; UAIM      0.001%; UAIO      0.001%; UCFI      0.259%;
 UCFM      0.445%; UCFO      0.312%; UCII      0.000%; UCIM      0.000%; UCIO      0.000%;
 UL01      0.022%; UL02      0.006%; UL03      0.226%; UL04      0.006%; UL05      0.012%;
 UL06      0.001%; UL07      0.002%; UL08      0.081%; UL09      0.034%; UL10      0.038%;
 UL11      0.013%; TRD1      0.003%; TRD2      0.004%; TRD3      0.000%; BR2_      0.018%;
 CB2_      0.011%; R1R2      0.018%; R2R2      0.898%; R3R2      0.003%; R3L2      0.001%;
 R1R1      0.414%; R2R1      0.010%; R3R1      0.009%; R1L1      0.036%; R3L1      0.001%;
 CA02      0.016%; CG02      0.000%; CA03      0.042%; CG03      0.000%; EMCA      0.013%;
 PTXW      0.013%; PUFP      0.005%; PTCB      0.008%; PPAP      0.005%; PXTL      0.072%;
 PASP      0.026%; MPPS      0.004%; UAPP      0.000%; LCPP      0.037%; DW11      0.000%;
 DV11      0.000%; DPPB      0.659%; DPFE      0.157%; DPMD      0.000%; DIQU      0.016%;
***************************************************************************
root [1] 
\end{verbatim}

At this point a file called \texttt{galice.root} is created in your present
directory. This file contains all of the ``hits'' produced by the simulation,
all of the particle information, all of the detectors that were defined in the
simulation, and a lot of other information.