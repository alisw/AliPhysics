\documentclass[11pt]{article}
\usepackage[margin=2cm,twoside,a4paper]{geometry}
\usepackage{amstext}
\usepackage{amsmath}
\usepackage[ruled,vlined,linesnumbered]{algorithm2e}
\usepackage{graphicx}
\usepackage{color}
\usepackage{units}
\usepackage{listings}
\usepackage[colorlinks,urlcolor=black,hyperindex,%
            linktocpage,a4paper,bookmarks=true]{hyperref}
\def\AlwaysText#1{\ifmmode\relax\text{#1}\else #1\fi}
\newcommand{\AbbrName}[1]{\AlwaysText{{\scshape #1}}}
\newcommand{\CERN}{\AbbrName{cern}}
\newcommand{\ALICE}{\AbbrName{alice}}
\newcommand{\SPD}{\AbbrName{spd}}
\newcommand{\ESD}{\AbbrName{esd}}
\newcommand{\AOD}{\AbbrName{aod}}
\newcommand{\INEL}{\AbbrName{inel}}
\newcommand{\INELONE}{$\AbbrName{inel}>0$}
\newcommand{\NSD}{\AbbrName{nsd}}
\newcommand{\FMD}[1][]{\AbbrName{fmd\ifx|#1|\else#1\fi}}
\newcommand{\OCDB}{\AbbrName{ocdb}}
\newcommand{\mult}[1][]{\ensuremath N_{\text{ch}#1}}
\newcommand{\dndetadphi}[1][]{{\ensuremath% 
    \ifx|#1|\else\left.\fi%
    \frac{d^2\mult{}}{d\eta\,d\varphi}%
    \ifx|#1|\else\right|_{#1}\fi%
}}
\newcommand{\landau}[1]{{\ensuremath% 
    \text{landau}\left(#1\right)}}
\newcommand{\dndeta}[1][]{{\ensuremath% 
    \ifx|#1|\else\left.\fi%
    \frac{1}{N}\frac{d\mult{}}{d\eta}%
    \ifx|#1|\else\right|_{#1}\fi%
}}
\newcommand{\MC}{\AlwaysText{MC}}
\newcommand{\Ntrgvtx}[1][]{{\ensuremath N_{#1\text{trg,vtx}}}}
\newcommand{\Ntrg}{{\ensuremath N_{\text{trg}}}}
\newcommand{\Nvtx}[1][]{{\ensuremath N_{#1\text{vtx}}}}
\newcommand{\Nsel}{{\ensuremath N_{\text{selected}}}}
\newcommand{\Ngood}{{\ensuremath N_{\text{good}}}}
\newcommand{\GeV}[1]{\unit[#1]{\AlwaysText{GeV}}}
\newcommand{\cm}[1]{\unit[#1]{\AlwaysText{cm}}}
\newcommand{\secref}[1]{Section~\ref{#1}}
\newcommand{\figref}[1]{Figure~\ref{#1}}
\newcommand{\etaphi}{\ensuremath(\eta,\varphi)}

\setlength{\parskip}{1ex}
\setlength{\parindent}{0em}

\title{Analysing the FMD data for $\dndeta$}
\author{Christian Holm
  Christensen\thanks{\texttt{$\langle$cholm@nbi.dk$\rangle$}}\quad\&\quad
  Hans Hjersing Dalsgaard\thanks{\texttt{$\langle$canute@nbi.dk$\rangle$}}\\ 
  Niels Bohr Institute\\
  University of Copenhagen}
\date{\today}
\begin{document}
\maketitle 

\tableofcontents 
\section{Introduction}

This document describes the steps performed in the analysis of the
charged particle multiplicity in the forward pseudo--rapidity
regions.  The primary detector used for this is the \FMD{}
\cite{FWD:2004mz,cholm:2009}.  The \SPD{} is used for determination of
the position of the primary interaction point.  

The analysis is performed as a two--step process.  
\begin{enumerate}
\item The Event--Summary--Data (\ESD{}) is processed event--by--event
  and passed through a number of algorithms, and
  $\dndetadphi$ for each event is output to an Analysis--Object--Data
  (\AOD{}) tree.
\item The \AOD{} data is read in and the sub--sample of the data under
  investigation is selected (e.g., \INEL{}, \INELONE{}, \NSD{}, or
  some centrality class) and the $\dndetadphi$ histogram read in for
  those events to build up $\dndeta$
\end{enumerate}
The details of each step above will be expanded upon in the
following. 

\section{Generating $\dndetadphi[i]$ event--by--event}

When reading in the \ESD{}s and generating the $\dndetadphi$
event--by--event the following steps are taken (in order) for each
event $i$
\begin{description}
\item[Event inspection] The global properties of the event is
  determined, including the trigger type and primary interaction
  point\footnote{`Vertex' and `primary interaction point' will be used
    interchangeably in the text, since there is no ambiguity with
    particle production vertex in this analysis.} $z$ coordinate (see
  \secref{sec:sub:event_inspection}).  
\item[Sharing filter] The \ESD{} object is read in and corrected for
  sharing.  The result is a new \ESD{} object (see
  \secref{sec:sub:sharing_filter}). 
\item[Density calculator] The (possibly un--corrected) \ESD{} object
  is then inspected and an inclusive (primary \emph{and} secondary
  particles), per--ring charged particle density
  $\dndetadphi[incl,r,v,i]$ is made.  This calculation depends in
  general upon the interaction vertex position along the $z$ axis
  $v_z$ (see \secref{sec:sub:density_calculator}).
\item[Corrections] The 5 $\dndetadphi[incl,r,v,i]$ are corrected for
  secondary production and acceptance.  The correction for the
  secondary particle production is highly dependent on the vertex $z$
  coordinate.  The result is a per--ring, charged primary particle
  density $\dndetadphi[r,v,i]$ (see \secref{sec:sub:corrector}). 
\item[Histogram collector] Finally, the 5 $\dndetadphi[r,v,i]$ are
  summed into a single $\dndetadphi[v,i]$ histogram, taking care of
  the overlaps between the detector rings.  In principle, this
  histogram is independent of the vertex, except that the
  pseudo--rapidity range, and possible holes in that range, depends on
  $v_z$ --- or rather the bin in which the $v_z$ falls (see
  \secref{sec:sub:hist_collector}).
\end{description}

Each of these steps will be detailed in the following. 

\subsection{Event inspection}
\label{sec:sub:event_inspection}

The first thing to do, is to inspect the event for triggers.  A number
of trigger bits, like \INEL{}, \INELONE{}, \NSD{}, and so on is then
propagated to the \AOD{} output.  

Just after the sharing filter (described below) but before any more
processing, the vertex information is queried.  If there is no vertex
information, or if the vertex $z$ coordinate is outside the
pre--defined range, then no further processing of that event takes place. 

\subsection{Sharing filter}
\label{sec:sub:sharing_filter}

The \FMD{} \ESD{} object contains the scaled energy deposited $\Delta
E/\Delta E_{mip}$ for each of the 51,200 strips.  The \FMD{} is
organised in 3 \emph{sub--detectors} \FMD{1}, \FMD{2}, and \FMD{3}, each
consisting of 1 (\FMD{1}) or 2 (\FMD{2} and \FMD{3}) \emph{rings}.
The rings fall into two types: \emph{Inner} or \emph{outer} rings.
Each ring is in turn  azimuthal divided into \emph{sectors}, and each
sector is radially divided into \emph{strips}.  How many sectors,
strips, as well as the $\eta$ coverage is given in
\tablename~\ref{tab:fmd:overview}. 

\begin{table}[htbp]
  \begin{center}
    \caption{Physical dimensions of Si segments and strips.}
    \label{tab:fmd:overview}
    \vglue0.2cm
    \begin{tabular}{|c|cc|cr@{\space--\space}l|r@{\space--\space}l|}
      \hline
      \textbf{Sub--detector/} &
      \textbf{Azimuthal}&
      \textbf{Radial} &
      $z$ &
      \multicolumn{2}{c|}{\textbf{$r$}} &
      \multicolumn{2}{c|}{\textbf{$\eta$}} \\
      \textbf{Ring}& 
      \textbf{sectors} &
      \textbf{strips} & 
      \textbf{[cm]} &
      \multicolumn{2}{c|}{\textbf{range [cm]}} &
      \multicolumn{2}{c|}{\textbf{coverage}} \\
      \hline
      FMD1i & 20& 512& 320  &  4.2& 17.2& 3.68&  5.03\\
      FMD2i & 20& 512&  83.4&  4.2& 17.2& 2.28&  3.68\\
      FMD2o & 40& 256&  75.2& 15.4& 28.4& 1.70&  2.29\\
      FMD3i & 20& 512& -75.2&  4.2& 17.2&-2.29& -1.70\\
      FMD3o & 40& 256& -83.4& 15.4& 28.4&-3.40& -2.01\\
      \hline
    \end{tabular}
  \end{center}
\end{table}

A particle originating from the vertex can, because of it's incident
angle on the \FMD{} sensors traverse more than one strip (see
\figref{fig:share_fraction}).  This means that the energy loss of the
particle is distributed over 1 or more strips.  The signal in each
strip should therefore possibly be merged with it's neighboring strip
signals to properly reconstruct the energy loss of a single particle.

\begin{figure}[htbp]
  \centering
  \includegraphics[keepaspectratio,height=3cm]{share_fraction}
  \caption{A particle traversing 2 strips and depositing energy in
    each strip. }
  \label{fig:share_fraction}
\end{figure}

The effect is most pronounced in low--flux\footnote{Events with a low
  hit density.} events, like proton--proton collisions or peripheral
Pb--Pb collisions, while in high--flux events the hit density is so
high that most likely each and every strip will be hit and the effect
cancel out on average.

Since the particles travel more or less in straight lines toward the
\FMD{} sensors, the sharing effect predominantly in the $r$ or
\emph{strip} direction.  Only neighboring strips in a given sector is
therefor investigated for this effect.  

Algorithm~\ref{algo:sharing} is applied to the signals in a given
sector.

\begin{algorithm}[htpb]
  \SetKwData{usedThis}{current strip used}
  \SetKwData{usedPrev}{previous strip used}
  \SetKwData{Output}{output}
  \SetKwData{Input}{input}
  \SetKwData{Nstr}{\# strips}
  \SetKwData{Signal}{current}
  \SetKwData{Eta}{$\eta$}
  \SetKwData{prevE}{previous strip signal} 
  \SetKwData{nextE}{next strip signal} 
  \SetKwData{lowFlux}{low flux flag} 
  \SetKwFunction{SignalInStrip}{SignalInStrip}
  \SetKwFunction{MultiplicityOfStrip}{MultiplicityOfStrip}
  \usedThis $\leftarrow$ false\;
  \usedPrev $\leftarrow$ false\;
  \For{$t\leftarrow1$ \KwTo \Nstr}{ 
    \Output${}_t\leftarrow 0$\;
    \Signal $\leftarrow$ \SignalInStrip($t$)\;

    \uIf{\Signal is not valid}{ 
      \Output${}_t \leftarrow$ invalid\;
    }
    \uElseIf{\Signal is 0}{ 
      \Output${}_t \leftarrow$ 0\;
    }
    \Else{
      \Eta$\leftarrow$ $\eta$ of \Input${}_t$\;
      \prevE$\leftarrow$ 0\;
      \nextE$\leftarrow$ 0\;
      \lIf{$t \ne 1$}{ 
        \prevE$\leftarrow$ \SignalInStrip($t-1$)\;
      }
      \lIf{$t \ne $\Nstr}{ 
        \nextE$\leftarrow$ \SignalInStrip($t+1$)\;
      }
      \Output${}_t\leftarrow$
      \MultiplicityOfStrip(\Signal,\Eta,\prevE,\nextE,\\
      \hfill\lowFlux,$t$,\usedPrev,\usedThis)\;
    }   
  }
  \caption{Sharing correction}
  \label{algo:sharing}
\end{algorithm}

Here the function \FuncSty{SignalInStrip}($t$) returns the properly
path--length corrected signal in strip $t$.  The function
\FuncSty{MultiplicityOfStrip} is where the real processing takes
place (see page \pageref{func:MultiplicityOfStrip}). 

\begin{function}[htbp]
  \caption{MultiplicityOfStrip(\DataSty{current},$\eta$,\DataSty{previous},\DataSty{next},\DataSty{low
      flux flag},\DataSty{previous signal used},\DataSty{this signal
      used})} 
  \label{func:MultiplicityOfStrip}
  \SetKwData{Current}{current} 
  \SetKwData{Next}{next} 
  \SetKwData{Previous}{previous} 
  \SetKwData{lowFlux}{low flux flag}
  \SetKwData{usedPrev}{previous signal used}
  \SetKwData{usedThis}{this signal used}
  \SetKwData{lowCut}{low cut}
  \SetKwData{total}{Total}
  \SetKwData{highCut}{high cut}
  \SetKwData{Eta}{$\eta$}  
  \SetKwFunction{GetHighCut}{GetHighCut}
  \If{\Current is very large or \Current $<$ \lowCut} {
    \usedThis $\leftarrow$ false\;
    \usedPrev $\leftarrow$ false\;
    \Return{0}
  }
  \If{\usedThis}{ 
    \usedThis $\leftarrow$ false\;
    \usedPrev $\leftarrow$ true\;
    \Return{0}
  }
  \highCut $\leftarrow$ \GetHighCut($t$,\Eta)\;
  \If{\Current $<$ \Next and \Next $>$ \highCut and \lowFlux set}{ 
    \usedThis $\leftarrow$ false\;
    \usedPrev $\leftarrow$ false\;
    \Return{0}
  }
  \total $\leftarrow$ \Current\;
  \lIf{\lowCut $<$ \Previous $<$ \highCut and not \usedPrev}{ 
    \total $\leftarrow$ \total + \Previous\;
  }
  \If{\lowCut $<$ \Next $<$ \highCut}{ 
    \total $\leftarrow$ \total + \Next\;  
    \usedThis $\leftarrow$ true\;
  }
  \eIf{\total $>$ 0}{ 
    \usedPrev $\leftarrow$ true\;
    \Return{\total}
  }{
    \usedPrev $\leftarrow$ false\;
    \usedThis $\leftarrow$ false\;
    \Return{0}
  }
\end{function}
Here, the function \FuncSty{GetHighCut} evaluates a fit to the energy
distribution in the specified $\eta$ bin (see also
\secref{sec:sub:density_calculator}).  It returns
$$
\Delta_{mp} - 2 w
$$
where $\Delta_{mp}$ is the most probable energy loss, and $w$ is the
width of the Landau distribution.  

The \KwSty{if} in line 5, says that if the previous strip was merged
with current one, and the signal of the current strip was added to
that, then the current signal is set to 0, and we mark it as used for
the next iteration (\DataSty{previous signal used}$\leftarrow$true).

The \KwSty{if} in line 10 checks if the current signal is smaller than
the next signal, if the next signal is larger than the upper cut
defined above, and if we have a low--flux event\footnote{Note, that in
  the current implementation there are never any low--flux events.}.
If that condition is met, then the current signal is the smaller of
two possible candidates for merging, and it should be merged into the
next signal.  Note, that this \emph{only} applies in low--flux events.

In line 15, we test if the previous signal lies between our low and
high cuts, and if it has not been marked as being used.  If so, we add
it to our current signal.  

The next \KwSty{if} on line 16 checks if the next signal is within our
cut bounds.  If so, we add that signal to the current signal and mark
it as used for the next iteration (\DataSty{this signal
  used}$\leftarrow$true).  It will then be zero'ed on the next
iteration by the condition on line 6.

Finally, if our signal is still larger than 0, we return the signal
and mark this signal as used (\DataSty{previous signal
  used}$\leftarrow$true) so that it will not be used in the next
iteration. Otherwise, we mark the current signal and the next signal
as unused and return a 0. 


\subsection{Density calculator}
\label{sec:sub:density_calculator}

The density calculator loops over all the strip signals in the \ESD{}
and calculates the inclusive (primary + secondary) charged particle
density in pre--defined $\etaphi$ bins.

\subsubsection{Inclusive number of charged particles} 

The number charged particles in a strip is calculated using multiple
Landau-like distributions fitted to the energy loss spectrum at a given
$\eta$ value.
\begin{align}
  \Delta_{i,mp} &= i (\Delta_{1,mp}+ \xi_1 \log(i))\nonumber\\
  \xi_i         &= i\xi_1\nonumber\\
  \sigma_i      &= \sqrt{i}\sigma_1\nonumber\\
  \mult[,t]     &= \frac{\sum_i^{N_{max}}
    i\,a_i\,F(\Delta_t;\Delta_{i,mp},\xi_i,\sigma_i)}{
    \sum_i^{N_{max}}\,a_i\,F(\Delta_t;\Delta_{i,mp},\xi_i,\sigma_i)}\quad,
\end{align}
where $F(x;\Delta_{mp},\xi,\sigma)$ is the evaluation of the Landau
distribution $f_L$ with most probable value $\Delta_{mp}$ and width
$\xi$, folded with a Gaussian distribution with spread $\sigma$ at the
energy loss $x$ \cite{nim:b1:16,phyrev:a28:615}.
\begin{align}
  \label{eq:energy_response}
  F(x;\Delta_{mp},\xi,\sigma) = \frac{1}{\sigma \sqrt{2 \pi}}
  \int_{-\infty}^{+\infty} d\Delta' f_{L}(x;\Delta',\xi)
  \exp{-\frac{(\Delta_{mp}-\Delta')^2}{2\sigma^2}}\quad,
\end{align}
where $\Delta_{1,mp}$, $\xi_1$, and $\sigma_1$ are the parameters for
the first MIP peak, $a_1=1$, and $a_i$ is the relative weight of the
$i^{\text{th}}$ MIP peak.  The parameters $\Delta_{1,mp}, \xi_1,
\sigma_1, a_2, \ldots a_{N_{max}}$ are obtained by fitting
$$
F_j(x;\Delta_{mp},\xi,\sigma) = 
\sum_{i=1}^{j} F(x;\Delta_{i,mp},\xi_{i},\sigma_i) 
$$
for increasing $j$ to the energy loss spectra in separate $\eta$ bins.
The fit procedure is stopped when the reduced $\chi^2$ exceeds a
certain threshold, or when the weight $a_j$ is smaller than some
number (typically $10^5$).  An example of the result of these fits are
given in \figref{fig:eloss_fits} in Appendix~\ref{app:eloss_fits}. 

\subsubsection{Azimuthal Acceptance}

Before the signal $\mult[,t]$ can be added to the $\etaphi$
bin in one of the 5 per--ring histograms, it needs to be corrected for
the $\varphi$ acceptance of the strip.

The acceptance correction is only applicable where the strip length
does not cover the full sector.  This is the case for the outer strips
in both the inner and outer type rings.  The acceptance correction is
then simply 
\begin{align}
  \label{eq:acc_corr}
  a_t &= \frac{l_t}{\Delta\varphi}\quad
\end{align}
where $l_t$ is the strip length in radians at constant $r$, and
$\Delta\varphi$ is $2\pi$ divided by the number of sectors in the
ring (20 for inner type rings, and 40 for outer type rings). 

The final $\etaphi$ content of the 5 output vertex dependent,
per--ring histograms of the inclusive charged particle density is then
given by
\begin{align}
  \label{eq:density}
  \dndetadphi[incl,r,v,i\etaphi] &= \sum_t^{t\in\etaphi}
  \mult[,t]\,a_t
\end{align}
where $t$ runs over the strips in the $\etaphi$ bin. 

\subsection{Corrections}
\label{sec:sub:corrector}

The corrections code receives the five vertex dependent,
per--ring histograms of the inclusive charged particle density
$\dndetadphi[incl,r,v,i]$ from the density calculator and applies
two corrections 

\subsubsection{Secondary correction}
%%
%%                hHits_FMD<d><r>_vtx<v> 
%% hCorrection = -----------------------
%%                hPrimary_FMD_<r>_vtx<v>
%%
%% where 
%% - hPrimary_FMD_<r>_vtx<vtx> is 2D of eta,phi for all primary ch
%%   particles
%% - hHits_FMD<d><r>_vtx<v>  is 2D of eta,phi for all track-refs that
%%   hit the FMD - The 2D version of hMCHits_nocuts_FMD<d><r>_vtx<v>
%%   used below. 
This is a 2 dimensional histogram generated from simulations, as the
ratio of primary particles to the total number of particles that fall
within an $\etaphi$ bin for a given vertex bin

\begin{align}
  \label{eq:secondary}
  s_v\etaphi &=
  \frac{\sum_i^{\Ntrgvtx[v,]}\mult[,\text{primary},i]\etaphi}{
    \sum_i^{\Ntrgvtx[v,]}\mult[,\text{\FMD{}},i]\etaphi}\quad,
\end{align}
where $\Ntrgvtx$ is the number of events with a valid trigger and a
vertex in bin $v$, and $\mult[,\FMD{},i]$ is the total number of
charged particles that hit the \FMD{} in event $i$ in the specified
$\etaphi$ bin and $\mult[,\text{primary},i]$ is number of
primary charged particles in event $i$ within the specified
$\etaphi$ bin.

$\mult[,\text{primary}]\etaphi$ is given by summing over the
charged particles labelled as primaries \emph{at the time of the
  collision} as defined in the simulation code.  That is, it is the
number of primaries within the $\etaphi$ bin at the collision
point --- not at the \FMD{}.

\subsubsection{Acceptance due to dead channels}

Some of the strips in the \FMD{} have been marked up as \emph{dead},
meaning that they are not used in the reconstruction or analysis.
This leaves holes in the acceptance of each defined $\etaphi$
which need to be corrected for.  

Dead channels are marked specially in the \ESD{}s with the flag
\textit{Invalid Multiplicity}.  This is used in the analysis to build
up and event--by--event acceptance correction in each $\etaphi$
bin by calculating the ratio
\begin{align}
  \label{eq:dead_channels} 
  a_{v,i}\etaphi &= 
  \frac{\sum_t^{t\in\etaphi}\left\{\begin{array}{cl}
        1 & \text{if not dead}\\
        0 & \text{otherwise}
      \end{array}\right.}{\sum_t^{t\in\etaphi} 1}\quad,
\end{align}
where $t$ runs over the strips in the $\etaphi$ bin.  This
correction is obviously $v_z$ dependent since which $\etaphi$
bin a strip $t$ corresponds to depends on it's relative position to
the primary vertex. 

Alternatively, pre--made acceptance factors can be used.  These are
made from the off-line conditions database (\OCDB{}).

The 5 output vertex dependent, per--ring histograms of the primary
charged particle density is then given by
\begin{align}
  \dndetadphi[r,v,i\etaphi] &=
  s_v\etaphi a_{v,i}\etaphi\dndetadphi[incl,r,v,i\etaphi]
\end{align}

\subsection{Histogram collector}
\label{sec:sub:hist_collector}

The histogram collector collects the information from the 5 vertex
dependent, per--ring histograms of the primary charged particle
density $\dndetadphi[r,v,i]$ into a single vertex dependent histogram
of the charged particle density $\dndetadphi[v,i]$.  

To do this, it first calculates, for each vertex bin, the $\eta$ bin
range to use for each ring.  It investigates the secondary correction
maps $s_v\etaphi$ to find the edges of the map.  The edges are
given by the $\eta$ range where $s_v\etaphi$ is larger than
some threshold\footnote{Typically $t_s\approx 0.1$.}  $t_s$. The code
applies safety margin of a $N_{cut}$ bins\footnote{Typically
  $N_{cut}=1$.}, to ensure that the data selected does not have too
large corrections associated with it.

It then loops over the bins in the defined $\eta$ range and sums the
contributions from each of the 5 histograms.  In the $\eta$ ranges
where two rings overlap, the collector calculates the average and adds
the errors in quadrature.

The output vertex dependent histogram of the primary
charged particle density is then given by
\begin{align}
  \label{eq:superhist}
  \dndetadphi[v,i\etaphi] &=
  \frac{1}{N_{r\in\etaphi}}\sum_{r}^{r\in\etaphi}  
  \dndetadphi[r,v,i\etaphi]\\
  \delta\left[\dndetadphi[v,i\etaphi]\right] &=
  \frac{1}{N_{r\in\etaphi}}\sqrt{\sum_{r}^{r\in\etaphi}   
    \delta\left[\dndetadphi[r,v,i\etaphi]\right]^2}
  \quad,
\end{align}
where $N_{r\in\etaphi}$ is the number of overlapping histograms
in the given $\etaphi$ bin. 

The histogram collector stores the found $\eta$ ranges in the
underflow bin of the histogram produced.  The content of the overflow
bins are 
\begin{align}
  \label{eq:overflow}
  I_{v,i}(\eta) &= 
  \frac{1}{N_{r\in(\eta)}}
  \sum_{r}^{r\in(\eta)} \left\{\begin{array}{cl} 
      0 & \eta \text{\ bin not selected}\\ 
      1 & \eta \text{\ bin selected}
      \end{array}\right.\quad,
\end{align}
where $N_{r\in(\eta)}$ is the number of overlapping histograms in the
given $\eta$ bin.  The subscript $v$ indicates that the content
depends on the current vertex bin of event $i$.

\section{Building the final $\dndeta$}

To build the final $\dndeta$ distribution it is enough to sum
\eqref{eq:superhist} and \eqref{eq:overflow} over all interesting
events and then normalise the selected trigger type.
\begin{align}
  \dndetadphi[\etaphi] &= \sum_i^{\Nsel}\dndetadphi[i,v\etaphi]\\ 
  I(\eta) &= \sum_i^{\Nsel}I_{i,v}(\eta)\quad.
\end{align}
Note, that $I(\eta)\le\Nsel$.  

With $N_x$ equal to the number of events with trigger $x$ (\INEL,
\INELONE, \NSD, or a particular centrality), we get
\begin{align}
  \label{eq:dndeta:x}
  \dndeta[(\eta),x] &=
  \frac{N_{\INEL}}{\Ngood}\frac{\Nsel}{N_x}\frac{1}{\Delta\eta}
  \frac{1}{I(\eta)}\sum_{\varphi}\dndetadphi[\etaphi]\quad,
\end{align}
where $\Nsel$ is the number of events with trigger $x$ and a vertex
within the define cuts, $\Ngood=N_B-N_A-N_C+2N_E$, $N_{\INEL}$ is the
number of inelastic triggers, and $\Delta\eta$ is the $\eta$ bin
width.  Note, that if $x=\INEL$ then the expression
\eqref{eq:dndeta:x} reduces to 
\begin{align*}
  \dndeta[(\eta),\INEL] =
  \frac{\Nsel}{\Ngood}\frac{1}{\Delta\eta}
  \frac{1}{I(\eta)}\sum_{\varphi}\dndetadphi[\etaphi]\quad.  
\end{align*}
Example code is shown in Appendix~\ref{app:exa_pass2}.

\section{Using the per--event $\dndetadphi[i,v]$ histogram for other
  analysis} 

\subsection{Multiplicity distribution} 

To build the multiplicity distribution for a given $\eta$ range
$[\eta_1,\eta_2]$, one needs to find the total multiplicity in that
$\eta$ range for each event. To do so, one should sum the
$\dndetadphi[i,v]$ histogram over all $\varphi$ and in the selected
$\eta$ range.
\begin{align}
  n'_{i[\eta_1,\eta_2]}, &= \int_{\eta_1}^{\eta_2}d\eta\int_0^{2\pi}d\varphi
  \dndetadphi[i,v]\quad.\nonumber
\end{align}
However, $n'_i$ is not corrected for the coverage in $\eta$ for the
particular vertex range $v$.  One therefor needs to correct for the
number of missing bins in the range $[\eta_1,\eta_2]$.  Suppose
$[\eta_1,\eta_2]$ covers $N_{[\eta_1,\eta_2]}$ $\eta$ bins, then the acceptance
correction is given by 
\begin{align}
  A_{i,[\eta_1,\eta_2]} = \frac{N_{[\eta_1,\eta_2]}}{\int_{\eta_1}^{\eta_2}d\eta\,
    I_{i,v}(\eta)}\quad.\nonumber
\end{align}
The per--event multiplicity is then given by 
\begin{align}
  n_{i,[\eta_1,\eta_2]} &= A_{i,[\eta_1,\eta_2]}\,n'_{i,[\eta_1,\eta_2]}\nonumber\\
  &= \frac{N_{[\eta_1,\eta_2]}}{\int_{\eta_1}^{\eta_2}\eta
    I_{i,v}(\eta)} \int_{\eta_1}^{\eta_2}d\eta\int_0^{2\pi}d\varphi
  \dndetadphi[i,v]
  \label{eq:event_n}
\end{align}

\subsection{Forward--Backward correlations} 

To do forward--backward correlations, one need to calculate
$n_{i,[\eta_1,\eta_2]}$ as shown in \eqref{eq:event_n} in two bins
$n_{i,[\eta_1,\eta_2]}$ and $n_{i,[-\eta_2,-\eta_1]}$ \textit{e.g.},
$n_{i,f}=n_{i,[-3,-1]}$ and $n_{i,b}=n_{i,[1,3]}$. 

\section{Some results}

\figurename{}s \ref{fig:1} to \ref{fig:3} shows some results.

\begin{figure}[tbp]
  \centering
  \includegraphics[keepaspectratio,width=\textwidth]{%
    dndeta_0900GeV_m10-p10cm_rb05_inel}
  \caption{$\dndeta$ for pp for \INEL{} events at $\sqrt{s}=\GeV{900}$,
    $\cm{-10}\le v_z\le\cm{10}$, rebinned by a factor 5.  Comparisons
    to other measurements shown where applicable}
  \label{fig:1}
\end{figure} 
\begin{figure}[tbp]
  \centering
  \includegraphics[keepaspectratio,width=\textwidth]{%
    dndeta_0900GeV_m10-p10cm_rb05_inelgt0}
  \caption{$\dndeta$ for pp for \INELONE{} events at
    $\sqrt{s}=\GeV{900}$, $\cm{-10}\le v_z\le\cm{10}$, rebinned by a
    factor 5.  Comparisons to other measurements shown where
    applicable}
  \label{fig:2}
\end{figure} 
\begin{figure}[tbp]
  \centering
  \includegraphics[keepaspectratio,width=\textwidth]{%
    dndeta_0900GeV_m10-p10cm_rb05_nsd}
  \caption{$\dndeta$ for pp for \NSD{} events at $\sqrt{s}=\GeV{900}$,
    $\cm{-10}\le v_z\le\cm{10}$, rebinned by a factor 5.  Comparisons
    to other measurements shown where applicable}
  \label{fig:3}
\end{figure} 

\clearpage
\appendix 
\section{Nomenclature} 

\begin{table}[hbp]
  \centering
  \begin{tabular}[t]{|lp{.8\textwidth}|}
    \hline 
    \textbf{Symbol}&\textbf{Description}\\
    \hline 
    \INEL & In--elastic event\\ 
    \INELONE & In--elastic event with at least one tracklet in the
    \SPD{} in the region $-1\le\eta\le1$\\ 
    \NSD{} & Non--single--diffractive event.  Single diffractive
    events are events where one of the incident collision systems
    (proton or nucleus) is excited and radiates particles, but there
    is no other processes taking place\\ 
    \hline
    $\Ntrg{}$ & Number of events with a valid (minimum bias) trigger\\
    $\Ntrgvtx{}$ & Number of events with a valid (minimum bias) trigger
    \emph{and} a valid vertex within the selected vertex range.\\ 
    $\Nsel{}$ & Number of events selected for analysis in the second
    pass\\ 
    $\Ngood$ & The number of expected \INEL{} events, given by the
    formula $N_B-N_A-N_C+2N_E$, where each of $N_x$ is count the
    number of interaction triggers with requirements of beam from both
    side, on the A side, on the C side, or no beam, respectively.\\
    \hline
    $\mult{}$ & Charged particle multiplicity\\ 
    $\mult[,\text{primary}]$ & Primary charged particle multiplicity
    as given by simulations\\ 
    $\mult[,\text{\FMD{}}]$ & Number of charged particles that hit the
    \FMD{} as given by simulations\\ 
    $\mult[,t]$ & Number of charged particles in an \FMD{} strip as
    given by evaluating the energy response functions $F$\\ 
    \hline
    $F$ & Energy response function (see \eqref{eq:energy_response})\\
    $\Delta_{mp}$ & Most probably energy loss\\ 
    $\xi$ & `Width' parameter of a Landau distribution\\
    $\sigma$ & Variance of a Gaussian distribution\\ 
    $n_i$ & Relative weight of the $i$--fold MIP peak in the energy
    loss spectra.\\ 
    \hline
    $a_t$ & Azimuthal acceptance of strip $t$\\ 
    $s_v\etaphi$ & Secondary particle correction factor in
    $\etaphi$ for a given vertex bin $v$\\ 
    $a_{v,i}\etaphi$ & Acceptance in $\etaphi$ for a given
    vertex bin $v$\\ 
    \hline
    $\dndetadphi[incl,r,v,i]$ & Inclusive (primary \emph{and}
    secondary) charge particle density in event $i$ with vertex $v$,
    for \FMD{} ring $r$.\\ 
    $\dndetadphi[r,v,i]$ & Primary charged particle
    density in event $i$ with vertex $v$ for \FMD{} ring $r$. \\
    $\dndetadphi[v,i]$ & Primary charged particle density in event $i$
    with vertex $v$\\  
    $I_{v,i}(\eta)$ & $\eta$ acceptance of event $i$ with vertex $v$\\ 
    $I(\eta)$ & Integrated $\eta$ acceptance over $\Nsel$ events.
    Note, that this has a value of $\Nsel$ for $(\eta)$ bins where we
    have full coverage\\ 
    \hline 
  \end{tabular}
  \caption{Nomenclature used in this document}
  \label{tab:nomenclature}
\end{table}
\clearpage


\section{Second pass example code}
\label{app:exa_pass2}
\lstset{basicstyle=\small\ttfamily,% 
  keywordstyle=\color[rgb]{0.627,0.125,0.941}\bfseries,% 
  identifierstyle=\color[rgb]{0.133,0.545,0.133}\itshape,%
  commentstyle=\color[rgb]{0.698,0.133,0.133},%
  stringstyle=\color[rgb]{0.737,0.561,0.561},
  emph={TH2D,TFile,TTree,AliAODForwardMult},emphstyle=\color{blue},%
  emph={[2]dndeta,sum,norm},emphstyle={[2]\bfseries\underbar},%
  language=c++,%
}
\begin{lstlisting}[caption={Example 2\textsuperscript{nd} pass code to
    do $\dndeta$},label={lst:example},frame=single,captionpos=b]
void Analyse()
{ 
  gSystem->Load("libANALYSIS.so");      // Load analysis libraries
  gSystem->Load("libANALYSISalice.so"); // General ALICE stuff
  gSystem->Load("libPWG2forward2.so");  // Forward analysis code

  TH2D*              sum        = 0;                  // Summed hist
  TFile*             file       = TFile::Open("AliAODs.root","READ");
  TTree*             tree       = static_cast<TTree*>(file->Get("aodTree"));
  AliAODForwardMult* mult       = 0;                  // AOD object
  int                nInel      = 0;                  // # of INEL ev.
  int                nGood      = 0;                  // # of total INEL
  int                nSelected  = 0;                  // # of triggered ev.
  int                nTriggered = 0;                  // # of ev. w/vertex
  float              vzLow      = -10;                // Lower ip cut
  float              vzHigh     = +10;                // Upper ip cut
  int                mask       = AliAODForwardMult::kInel;// Trigger mask
  tree->SetBranchAddress("Forward", &forward);        // Set the address

  for (int i = 0; i < tree->GetEntries(); i++) { 
    // Read the i'th event 
    tree->GetEntry(i);

    // Create sum histogram on first event - to match binning to input
    if (!sum) 
      sum = static_cast<TH2D*>(mult->GetHistogram()->Clone("d2ndetadphi"));
    
    // Count triggers 
    if (mult->IsTriggerBits(AliForwardMult::kB))    nGood++;
    if (mult->IsTriggerBits(AliForwardMult::kA))    nGood--;
    if (mult->IsTriggerBits(AliForwardMult::kC))    nGood--;
    if (mult->IsTriggerBits(AliForwardMult::kE))    nGood += 2;
    if (mult->IsTriggerBits(AliForwardMult::kInel)) nInel++;

    // Other trigger/event requirements could be defined 
    if (!mult->IsTriggerBits(mask)) continue; 
    nTriggered++;

    // Check if we have vertex and select vertex range (in centimeters) 
    if (!mult->HasIpZ() || !mult->InRange(vzLow, vzHigh) continue; 
    nSelected++;

    // Add contribution from this event
    sum->Add(&(mult->GetHistogram()));
  }

  // Get acceptance normalisation from underflow bins 
  TH1D* norm   = sum->Projection("norm", 0, 1, "");
  // Project onto eta axis - _ignoring_underflow_bins_!
  TH1D* dndeta = sum->Projection("dndeta", 1, -1, "e");
  // Normalize to the acceptance, and scale by the vertex efficiency 
  dndeta->Divide(norm);
  dndeta->Scale(double(nInel * nSelected) / (nTriggered * nGood), "width");
  // And draw the result
  dndeta->Draw();
}
\end{lstlisting}

\section{$\Delta E$ fits} 
\label{app:eloss_fits}

\begin{figure}[htbp]
  \centering
  \includegraphics[keepaspectratio,width=.8\textwidth]{eloss_fits}
  \caption{Summary of energy loss fits}
  \label{fig:eloss_fits}
\end{figure}

\begin{thebibliography}{99}
\bibitem{FWD:2004mz} \ALICE{} Collaboration, Bearden, I.~G.\ \textit{et al}
  \textit{ALICE technical design report on forward detectors: FMD, T0
    and V0}, \CERN{}, 2004, CERN-LHCC-2004-025
\bibitem{cholm:2009} Christensen, C.~H., \textit{The ALICE Forward
    Multiplicity Detector --- From Design to Installation},
  Ph.D.~thesis, University of Copenhagen, 2009,
  \url{http://www.nbi.dk/~cholm/}. 
\bibitem{nim:b1:16} Nucl.Instrum.Meth.B1:16
\bibitem{phyrev:a28:615} Phys.Rev.A28:615
\end{thebibliography}
\end{document}

% Local Variables:
%   ispell-local-dictionary: "british"
% End:
%
% LocalWords:  tracklet diffractive
